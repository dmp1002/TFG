\apendice{Anexo de sostenibilización curricular}

\section{Introducción}
Como estudiante, en este anexo presento mi reflexión personal sobre los aspectos de sostenibilidad que he integrado en mi Trabajo de Fin de Grado.
Toda esta reflexión se ha desarrollado siguiendo las instrucciones dadas en el documento sobre la introducción de la sostenibilidad de la CRUE.~\cite{dirsos}

\subsection{Competencias de Sostenibilidad Adquiridas}
A lo largo de la realización de mi proyecto, he logrado desarrollar diversas competencias relacionadas con la sostenibilidad:

\begin{itemize}
    \item \textbf{Pensamiento sistémico:} Para la realización de este proyecto se ha tenido una perspectiva holística, analizando los impactos sociales, ambientales y económicos de las soluciones propuestas en mi trabajo. Gracias a esto he sido capaz de reconocer la importancia de la sostenibilidad y del pensamiento sistémico, ya que puede ayudar a que todos nos ayudemos en nuestros proyectos.
    \item \textbf{Anticipación y visión de futuro:} Gracias a pensar a largo plazo he diseñado la aplicación de forma que en un futuro se pueda mejorar y seguir desarrollando. Esto es algo muy importante para ahorrar recursos y el tiempo de otras personas
    \item \textbf{Responsabilidad ética y profesional:} A lo largo de la realización del trabajo se ha ido pensando de forma responsable cómo actuar y qué decisiones tomar a la hora de desarrollar la aplicación.
    \item \textbf{Colaboración interdisciplinar:} En este proyecto me he tenido que sumergir en una temática muy diferente a la que me dedico normalmente, ya que he tenido que aprender sobre temas de agricultura, fertilizantes y el funcionamiento de las imágenes hiperespectrales.
    \item \textbf{Comunicación y sensibilización:} En este proyecto se ha tratado el tema de reducir el uso de fertizantes y la importancia de cuidar el medio ambiente de esta formma.
\end{itemize}


\section{Aplicación de la Sostenibilidad al Trabajo de Fin de Grado}
En mi Trabajo de Fin de Grado, he logrado integrar los principios de sostenibilidad de la siguiente manera:
\begin{itemize}
    \item \textbf{Análisis del ciclo de vida:} A la hora del desarrollo del proyecto se ha pensado de qué forma se podrá utilizar en un futuro la aplicación y si en algú momento se quedará obsoleta y habrá que crear otra.
    \item \textbf{Diseño circular:} Se han utilizado elementos modulares y que se pueden volver a utiliar como las difrentes librerías de Python que se han añadido al proyecto y que son de uso libre.
    \item \textbf{Inclusión social:} See ha pensado en que la aplicación sea accesible para todo el mundo, por eso se ha elegido una licencia MIT, para que todo el mundo pueda reutilzar el proyecto sin ningún problema.
    \item \textbf{Mitigación y adaptación al cambio climático:} Al estar estudiando la forma de reducir el uso de fertilizantes se está contribuyendo a mitigar el cambio climático. Toda pequeña acción suma.
    \item \textbf{Educación para la sostenibilidad:} Toda persona que utilice este proyecto será capaz de entender la importancia de las imágenes hiperespectrales y como pueden contribuir a la sostenibilidad, como por ejemplo sus usos en la agricultura como en este caso, o en la detección de la frescura de productos vegetales sin destruirlos.
\end{itemize}


En resumen, a través de este Trabajo de Fin de Grado, he logrado adquirir y aplicar diversas competencias de sostenibilidad que me permitirán desarrollar soluciones más responsables y comprometidas con el bienestar de la sociedad y el medioambiente. El objetivo ha sido integrar los principios de sostenibilidad de manera transversal en todo el proyecto, ya sea en la documentación, en el código o en la realización de los Sprints,